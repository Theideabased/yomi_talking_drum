\documentclass[12pt,a4paper]{article}
\usepackage[utf8]{inputenc}
\usepackage{amsmath,amsfonts,amssymb}
\usepackage{graphicx}
\usepackage{geometry}
\usepackage{hyperref}
\usepackage{titlesec}
\usepackage{fancyhdr}
\usepackage{setspace}
\usepackage[round]{natbib}
\usepackage{times}
\usepackage{array}
\usepackage{booktabs}
\usepackage{longtable}

\geometry{margin=1in}
\doublespacing
\bibliographystyle{apalike}

\pagestyle{fancy}
\fancyhf{}
\rhead{Chapter 3 - Methodology}
\lhead{Developing Talking Drums Dataset for AI Patterns}
\cfoot{\thepage}

\title{\textbf{CHAPTER 3: METHODOLOGY}}
\author{}
\date{}

\begin{document}

\maketitle

\section{INTRODUCTION}

This chapter presents a comprehensive methodology for developing a talking drums dataset for artificial intelligence pattern generation. The approach systematically leverages existing digital audio resources while maintaining the highest standards of cultural authenticity and technical rigor, following established frameworks in music information retrieval \citep{muller2015music, lerch2012introduction} and ethnomusicological research \citep{nettl2005study, titon2017worlds}.

The methodology is designed to be both academically robust and practically implementable. Each technical process is explained in clear terms before presenting the mathematical foundation, ensuring accessibility to readers from diverse backgrounds while maintaining scholarly depth, as recommended by \citet{creswell2014research} for mixed-methods research approaches.

The research adopts a resource-integration approach that curates, processes, and synthesizes available materials from multiple sources. This practical strategy enables immediate research implementation while establishing scalable methodologies for future dataset expansion, aligning with current trends in digital humanities \citep{moretti2013distant, hockey2004history} and computational ethnomusicology \citep{serra2011roadmap, fuentes2007computational}.

\textbf{Mathematical Framework Overview:}

The entire methodology can be represented as an optimization problem where we seek to maximize dataset quality $Q(D)$, authenticity $A(D)$, and cultural preservation $C(D)$:

\begin{equation}
\text{Maximize: } \text{Objective Function} = Q(D) \times A(D) \times C(D)
\end{equation}

Subject to practical constraints:
\begin{align}
\text{Dataset size: } &|D| \geq 500 \text{ samples} \\
\text{Audio quality: } &\text{SNR} \geq 35 \text{ dB} \\
\text{Sampling frequency: } &f_s \geq 44.1 \text{ kHz} \\
\text{License compatibility: } &\text{Creative Commons or Academic Use}
\end{align}

This mathematical foundation guides every decision in the methodology, ensuring systematic and measurable progress toward our research goals.

\section{RESEARCH DESIGN AND APPROACH}

\subsection{Resource-Integration Framework}

Traditional ethnomusicological research typically involves extensive field recordings, which can be time-consuming and logistically challenging \citep{rice2014ethnomusicology, barz2008shadows}. Our resource-integration approach offers a practical alternative by systematically identifying, evaluating, and curating existing digital audio materials, following principles established in digital library science \citep{arms2000digital, borgman2015big}.

This framework operates on four fundamental principles:

\begin{enumerate}
\item \textbf{Systematic Resource Discovery}: We conduct comprehensive searches across digital platforms, academic repositories, and cultural archives. This process follows structured methodology that ensures complete coverage of available materials \citep{terras2015digitisation, zorich2012transitioning}.

\item \textbf{Quality-Driven Curation}: Every audio sample undergoes rigorous evaluation using both technical metrics and cultural authenticity criteria \citep{casey2008content, klapuri2006signal}.

\item \textbf{Standardized Processing}: All selected materials pass through a uniform preprocessing pipeline following established audio processing standards \citep{zolzer2011dafx, orfanidis2007introduction}.

\item \textbf{Comprehensive Annotation}: We implement multi-dimensional labeling systems based on established practices in music informatics \citep{downie2003music, schedl2014music}.
\end{enumerate}

\textbf{Mathematical Representation:}

Each resource $r_i$ is characterized by a feature vector:

\begin{equation}
r_i = [q_i, a_i, c_i, l_i, t_i]^T
\end{equation}

Where:
\begin{itemize}
\item $q_i$ = Quality score (0 to 1, where 1 is highest quality)
\item $a_i$ = Authenticity score (0 to 1, where 1 is most authentic)
\item $c_i$ = Cultural relevance score (0 to 1, where 1 is most relevant)
\item $l_i$ = License compatibility (0 or 1, binary indicator)
\item $t_i$ = Technical suitability score (positive real number)
\end{itemize}

The resource selection process uses a weighted scoring function:

\begin{equation}
\text{Selection Score} = 0.25 \times q_i + 0.30 \times a_i + 0.25 \times c_i + 0.10 \times l_i + 0.10 \times t_i
\end{equation}

Resources with scores $\geq 0.70$ are included in the dataset. This mathematical approach ensures objective, reproducible selection criteria while balancing all important factors.

\subsection{Theoretical Framework Integration}

Our methodology integrates three complementary theoretical approaches based on current research in computational musicology \citep{huron2006sweet, temperley2007music}:

\textbf{Digital Ethnomusicology Framework}

This approach utilizes digital tools and resources to study traditional music while maintaining cultural authenticity and respect for source communities \citep{seeger2004traditional, post2006ethnomusicology}. The mathematical model represents this as:

\begin{equation}
E = \{M, C, P, T\}
\end{equation}

Where $M$ = Musical analysis, $C$ = Cultural context, $P$ = Preservation techniques, $T$ = Traditional knowledge.

\textbf{Music Information Retrieval (MIR) Framework}

We apply computational methods specifically adapted for African musical contexts \citep{rycroft1961guitar, kubik1999africa}. The MIR framework is modeled as:

\begin{equation}
MIR = \{F, A, R, C\}
\end{equation}

Where: $F$ = Feature extraction, $A$ = Audio analysis, $R$ = Rhythm recognition, $C$ = Classification methods.

\textbf{Cross-Cultural AI Development Framework (CCAI):}

This ensures AI systems can understand and generate culturally appropriate content \citep{barocas2017fairness, liang2022holistic}:

\begin{equation}
CCAI = \{D, M, V, E\}
\end{equation}

Where: $D$ = Diverse datasets, $M$ = Multicultural models, $V$ = Cross-cultural validation, $E$ = Ethical considerations.

\section{SYSTEMATIC RESOURCE IDENTIFICATION AND ACQUISITION}

\subsection{Digital Platform Survey Strategy}

The resource discovery process follows a systematic approach to maximize the probability of finding high-quality talking drum samples, utilizing established methodologies in digital collection development \citep{lee2002collection, johnson2018fundamentals}.

Our primary focus centers on platforms with substantial African music collections:

\textbf{Freesound.org Community Repository}

Freesound.org represents our primary source for community-contributed samples \citep{font2013freesound}. The platform contains over 700,000 audio samples with Creative Commons licensing, making it ideal for research applications.

\textbf{Search Strategy Implementation:}

We employ a systematic search methodology using multiple keyword combinations:

\begin{itemize}
\item Primary keywords: ``talking drum'', ``dundun'', ``gangan'', ``yoruba drum''
\item Secondary keywords: ``tama'', ``sabar'', ``djembe'', ``west african drum''
\item Tertiary keywords: ``hourglass drum'', ``african percussion'', ``traditional drum''
\end{itemize}

The search effectiveness is measured mathematically as:

\begin{equation}
\text{Effectiveness} = \frac{|\text{Relevant Results} \cap \text{Retrieved Results}|}{|\text{Retrieved Results}|}
\end{equation}

\textbf{Expected Yield Calculation:}

Based on preliminary searches and quality filtering, we estimate:

\begin{equation}
\text{Total Expected Yield: } E[N] = \sum E[N_i] = 200-500 \text{ high-quality samples}
\end{equation}

\textbf{Academic Digital Archives}

Several academic institutions maintain digitized ethnomusicological collections relevant to this research \citep{seaman2011digital, proffitt2005preservation}:

\begin{itemize}
\item \textbf{British Library Sound Archive}: Comprehensive world music collection with professional-quality recordings
\item \textbf{Smithsonian Folkways Recordings}: Extensive African music catalog with detailed cultural documentation \citep{goldsmith2019smithsonian}
\item \textbf{Archive of World Music (Harvard)}: Academic collection with ethnomusicological context
\item \textbf{UCLA Ethnomusicology Archive}: Research-quality recordings with scholarly annotations
\end{itemize}

\subsection{Quality Assessment Framework}

Each audio sample undergoes comprehensive technical evaluation using established audio quality metrics \citep{thiede2000peaq, kabal2002examination}:

\begin{table}[h]
\centering
\caption{Audio Quality Assessment Metrics and Mathematical Formulations}
\begin{tabular}{@{}p{3cm}p{4cm}p{2cm}p{2cm}p{2cm}@{}}
\toprule
\textbf{Metric} & \textbf{Mathematical Formula} & \textbf{Minimum} & \textbf{Target} & \textbf{Weight} \\
\midrule
Signal-to-Noise Ratio & $\text{SNR}(\text{dB}) = 20 \times \log_{10}\left(\frac{\text{Signal Power}}{\text{Noise Power}}\right)$ & 35 dB & 50+ dB & 0.25 \\
\addlinespace
Dynamic Range & $\text{DR}(\text{dB}) = 20 \times \log_{10}\left(\frac{\text{Peak Amplitude}}{\text{RMS Amplitude}}\right)$ & 30 dB & 50+ dB & 0.20 \\
\addlinespace
Frequency Response & $\text{FR} = \frac{(f_{\max} - f_{\min})}{f_{\text{nyquist}}}$ & 0.68 & 0.95+ & 0.20 \\
\addlinespace
Total Harmonic Distortion & $\text{THD} = \sqrt{\sum \frac{\text{Harmonic Powers}}{\text{Fundamental Power}}}$ & $< 5\%$ & $< 1\%$ & 0.15 \\
\addlinespace
Spectral Bandwidth & $\text{BW} = \frac{\sum (\text{frequency} \times |\text{Spectrum}|^2)}{\sum |\text{Spectrum}|^2}$ & $> 2000$ Hz & $> 8000$ Hz & 0.20 \\
\bottomrule
\end{tabular}
\end{table}

The composite quality score combines these metrics:

\begin{equation}
Q_{\text{total}} = \sum_{i=1}^{5} w_i \times Q_i
\end{equation}

where $w_i$ represents the weight for metric $i$ and $Q_i$ is the normalized score for that metric.

\section{COMPREHENSIVE AUDIO PROCESSING AND ENHANCEMENT}

\subsection{Standardization and Preprocessing Pipeline}

All audio samples undergo systematic preprocessing following established digital audio standards \citep{pohlmann2005principles, watkinson2001art}:

\textbf{Technical Specifications:}
\begin{itemize}
\item Sample Rate: 44.1 kHz (CD quality standard)
\item Bit Depth: 24-bit (professional quality)
\item Format: WAV (uncompressed)
\item Channels: Mono (focused on rhythmic content)
\end{itemize}

\textbf{Processing Pipeline Mathematical Formulation:}

The preprocessing pipeline applies sequential transformations:

\begin{equation}
P(x) = T_5(T_4(T_3(T_2(T_1(x)))))
\end{equation}

Where:
\begin{align}
T_1(x) &= \text{Noise reduction using spectral subtraction} \\
T_2(x) &= \text{DC offset removal} \\
T_3(x) &= \text{Butterworth filtering} \\
T_4(x) &= \text{Dynamic range compression} \\
T_5(x) &= \text{Peak normalization}
\end{align}

\textbf{Noise Reduction Implementation:}

We employ spectral subtraction for noise reduction \citep{boll1979suppression, berouti1979enhancement}:

\begin{equation}
|\hat{S}(\omega)|^2 = |Y(\omega)|^2 - \alpha|\hat{N}(\omega)|^2
\end{equation}

Where $\hat{S}(\omega)$ is the estimated clean signal, $Y(\omega)$ is the noisy signal, and $\hat{N}(\omega)$ is the estimated noise spectrum.

\textbf{Dynamic Range Compression:}

The compression function follows standard audio processing practices \citep{katz2002mastering}:

\begin{equation}
y(n) = \begin{cases}
x(n), & \text{if } |x(n)| \leq T \\
T + \frac{|x(n)| - T}{R} \times \text{sign}(x(n)), & \text{if } |x(n)| > T
\end{cases}
\end{equation}

Where $T$ is the compression threshold and $R$ is the compression ratio.

\textbf{Normalization:}

Peak normalization ensures consistent amplitude levels:

\begin{equation}
x_{\text{norm}}(n) = \frac{x(n)}{\max(|x(n)|)} \times 0.95
\end{equation}

\subsection{Onset Detection and Segmentation}

Precise onset detection forms the foundation for rhythmic analysis, utilizing established algorithms in music information retrieval \citep{bello2005tutorial, dixon2006onset}.

\textbf{Multi-Method Onset Detection:}

We combine multiple onset detection methods for robustness:

\begin{align}
O_{\text{sf}}(n) &= \sum_k \max(0, |X(k,n)| - |X(k,n-1)|) \quad \text{(Spectral Flux)} \\
O_{\text{hfc}}(n) &= \sum_k k \times |X(k,n)|^2 \quad \text{(High-Frequency Content)} \\
O_{\text{cd}}(n) &= |\sum_k X(k,n) \times X^*(k,n-1)| \quad \text{(Complex Domain)}
\end{align}

The combined onset strength function is:

\begin{equation}
O_{\text{combined}}(n) = w_1 O_{\text{sf}}(n) + w_2 O_{\text{hfc}}(n) + w_3 O_{\text{cd}}(n)
\end{equation}

Weight optimization follows \citet{bello2005tutorial}:

\begin{equation}
w^* = \arg\min_w \sum_i \left(y_i - \sum_j w_j \times O_j(i)\right)^2
\end{equation}

\section{MULTI-DIMENSIONAL FEATURE EXTRACTION AND ANALYSIS}

\subsection{Time-Domain Feature Analysis}

Time-domain features capture temporal characteristics essential for understanding talking drum communication patterns \citep{mcaulay1986speech, quatieri2002discrete}:

\textbf{Root Mean Square Energy:}
\begin{equation}
\text{RMS}(n) = \sqrt{\frac{1}{N} \times \sum_{k=0}^{N-1} x^2(k)}
\end{equation}

\textbf{Zero-Crossing Rate:}
\begin{equation}
\text{ZCR}(n) = \frac{1}{2N} \times \sum_{k=0}^{N-1} |\text{sign}(x(k)) - \text{sign}(x(k-1))|
\end{equation}

\textbf{Attack and Decay Time Analysis:}

Following established practices in audio analysis \citep{peeters2004large}, attack time is calculated as:

\begin{equation}
\text{Attack Time} = \frac{(\text{Peak Index} - \text{Onset Index})}{\text{Sample Rate}}
\end{equation}

Decay time is similarly computed:

\begin{equation}
\text{Decay Time} = \frac{(\text{Fade Index} - \text{Peak Index})}{\text{Sample Rate}}
\end{equation}

\subsection{Frequency-Domain Analysis}

\textbf{Fundamental Frequency Estimation}

We employ the YIN algorithm \citep{de2002yin} for robust pitch detection:

\textbf{Difference Function:}
\begin{equation}
d_t(\tau) = \sum_{j=1}^W (x_j - x_{j+\tau})^2
\end{equation}

\textbf{Cumulative Mean Normalized Difference:}
\begin{equation}
d'_t(\tau) = \begin{cases}
1, & \text{if } \tau = 0 \\
\frac{d_t(\tau)}{\left(\frac{1}{\tau}\right) \times \sum_{k=1}^\tau d_t(k)}, & \text{otherwise}
\end{cases}
\end{equation}

\textbf{Fundamental Frequency:}
\begin{equation}
f_0 = \frac{f_s}{\tau_{\min}}
\end{equation}

where $\tau_{\min}$ is the first minimum of $d'_t(\tau)$ below the threshold.

\textbf{Spectral Features}

Following standard practices in music informatics \citep{tzanetakis2002musical}, we extract:

\textbf{Spectral Centroid:}
\begin{equation}
\text{SC} = \sum_k k \times \frac{|X(k)|^2}{\sum_k |X(k)|^2}
\end{equation}

\textbf{Spectral Rolloff:}
\begin{equation}
\text{SR} = k \text{ such that } \sum_{i=0}^k |X(i)|^2 = 0.85 \times \sum_{i=0}^{N-1} |X(i)|^2
\end{equation}

\textbf{Spectral Bandwidth:}
\begin{equation}
\text{SB} = \sqrt{\sum_k (k - \text{SC})^2 \times \frac{|X(k)|^2}{\sum_k |X(k)|^2}}
\end{equation}

\subsection{Rhythmic Feature Extraction}

\textbf{Mel-Frequency Cepstral Coefficients (MFCCs)}

Following \citet{logan2000mel}, MFCCs are computed using:

\begin{equation}
\text{mel}(f) = 2595 \times \log_{10}\left(1 + \frac{f}{700}\right)
\end{equation}

\textbf{Harmonic-Percussive Separation}

We utilize the method by \citet{fitzgerald2010harmonic} to separate harmonic and percussive components, with the ratio:

\begin{equation}
\text{H/P Ratio} = \frac{\text{Energy}_{(\text{Harmonic Component})}}{\text{Energy}_{(\text{Percussive Component})}}
\end{equation}

\textbf{Inter-Onset Interval Analysis}

For rhythm analysis, we compute inter-onset intervals \citep{london2012hearing}:

\begin{equation}
\text{IOI CV} = \frac{\text{Standard Deviation(IOI)}}{\text{Mean(IOI)}}
\end{equation}

\begin{equation}
\text{Regularity} = \frac{1}{(1 + \text{Standard Deviation(IOI)})}
\end{equation}

\textbf{Normalized Pairwise Variability Index (nPVI)}

Following \citet{grabe2002durational}:

\begin{equation}
\text{nPVI} = \frac{100}{m-1} \times \sum_{k=1}^{m-1} \frac{|\text{IOI}_k - \text{IOI}_{k+1}|}{(\text{IOI}_k + \text{IOI}_{k+1})}
\end{equation}

\textbf{Microtiming Deviation}

Timing accuracy relative to ideal grid positions:

\begin{equation}
\text{MTD} = \frac{1}{n} \times \sum_{i=1}^n |t_i - t_{\text{grid}}(i)|
\end{equation}

\subsection{Cultural and Linguistic Feature Analysis}

\textbf{Speech Surrogate Pattern Recognition}

This novel component analyzes drum-speech correspondence, building on work by \citet{carrington1949talking, gleick2011information}:

\textbf{Tonal Contour Analysis:}

Fundamental frequency is converted to semitones for tonal analysis:

\begin{equation}
F_{0_{\text{semitones}}}(n) = 12 \times \log_2\left(\frac{F_0(n)}{440}\right)
\end{equation}

\textbf{Tonal Movement:}
\begin{equation}
\text{TM} = \sum_{n=1}^{N-1} |F_{0_{\text{semitones}}}(n+1) - F_{0_{\text{semitones}}}(n)|
\end{equation}

\textbf{Direction Change Rate:}
\begin{equation}
\text{DCR} = \frac{\text{Number of Direction Changes}}{N-2}
\end{equation}

\textbf{Syllable Rate Analysis:}

Communication patterns are analyzed through syllable-like segmentation:

\begin{equation}
\text{Syllable Rate} = \frac{\text{Number of Syllables}}{\text{Total Duration}}
\end{equation}

\section{MACHINE LEARNING FRAMEWORK}

\subsection{Model Architecture Design}

Our generative framework utilizes state-of-the-art deep learning architectures, following recent advances in generative modeling \citep{goodfellow2016deep, kingma2013auto}.

\textbf{Cultural Distance Metric}

We define cultural similarity between samples using:

\begin{equation}
d_{\text{cultural}}(s_1, s_2) = \sqrt{\sum_{i=1}^d w_i (f_i(s_1) - f_i(s_2))^2}
\end{equation}

where $f_i$ represents cultural feature $i$ with weight $w_i$.

\textbf{Support Vector Machine Implementation}

For classification tasks, we employ SVM with RBF kernel \citep{cortes1995support}:

Optimization objective:
\begin{equation}
\text{Minimize: } \frac{1}{2}||w||^2 + C \sum_{i=1}^n \xi_i
\end{equation}

Subject to constraints following \citet{scholkopf2000new}.

\textbf{Deep Learning Architecture}

For generative modeling, we implement a multi-class neural network with cross-entropy loss:

\begin{equation}
L = -\frac{1}{n} \times \sum_{i=1}^n \sum_{j=1}^c y_{ij} \times \log(\hat{y}_{ij})
\end{equation}

\section{VALIDATION AND EVALUATION FRAMEWORK}

\subsection{Cross-Validation Strategy}

We employ stratified k-fold cross-validation following \citet{kohavi1995study}:

\begin{equation}
\text{CV Error} = \frac{1}{k} \times \sum_{i=1}^k L(f^{-i}, S_i)
\end{equation}

where $f^{-i}$ is the model trained excluding fold $i$, and $S_i$ is the test set for fold $i$.

\textbf{Confidence Intervals:}

Following standard statistical practices \citep{dietterich1998approximate}:

\begin{equation}
\mu_{\text{CV}} \pm t_{0.025,k-1} \times \frac{\sigma_{\text{CV}}}{\sqrt{k}}
\end{equation}

\subsection{Performance Metrics}

We evaluate model performance using established metrics in machine learning \citep{hastie2009elements, bishop2006pattern}:

\textbf{Classification Metrics:}
\begin{align}
\text{Accuracy} &= \frac{(\text{TP} + \text{TN})}{(\text{TP} + \text{TN} + \text{FP} + \text{FN})} \\
\text{Precision} &= \frac{\text{TP}}{\text{TP} + \text{FP}} \\
\text{Recall} &= \frac{\text{TP}}{(\text{TP} + \text{FN})} \\
\text{F1-Score} &= 2 \times \frac{\text{Precision} \times \text{Recall}}{\text{Precision} + \text{Recall}}
\end{align}

\textbf{Generation Quality Assessment}

For generative models, we employ Fréchet Audio Distance \citep{kilgour2018frechet}:

\begin{equation}
\text{FAD} = ||\mu_r - \mu_g||^2 + \text{Tr}\left(\Sigma_r + \Sigma_g - 2\sqrt{\Sigma_r \Sigma_g}\right)
\end{equation}

where $\mu_r, \Sigma_r$ and $\mu_g, \Sigma_g$ are the mean and covariance of real and generated samples respectively.

\section{ETHICAL CONSIDERATIONS AND CULTURAL SENSITIVITY}

Our research adheres to established ethical frameworks in cultural research \citep{brown2003ethics, seeger2004traditional}, ensuring:

\begin{itemize}
\item Respectful engagement with traditional knowledge systems
\item Proper attribution of cultural sources
\item Community benefit and knowledge sharing
\item Protection of sacred or sensitive cultural content
\end{itemize}

These considerations are integrated throughout the methodology, ensuring that technological advancement serves cultural preservation and understanding.

\section{IMPLEMENTATION TIMELINE AND RESOURCE ALLOCATION}

The project follows a structured 24-week implementation schedule:

\begin{table}[h]
\centering
\caption{Project Implementation Timeline}
\begin{tabular}{@{}p{2cm}p{2cm}p{4cm}p{3cm}p{3cm}@{}}
\toprule
\textbf{Phase} & \textbf{Duration} & \textbf{Key Activities} & \textbf{Mathematical Models} & \textbf{Expected Deliverables} \\
\midrule
Phase 1 & Weeks 1-6 & Resource Collection & $Q(D)$ optimization & Curated dataset (500+ samples) \\
Phase 2 & Weeks 7-12 & Processing \& Analysis & Feature extraction pipeline & Processed audio + features \\
Phase 3 & Weeks 13-18 & Model Development & ML/DL architectures & Trained models \\
Phase 4 & Weeks 19-24 & Validation \& Testing & Performance metrics & Validated system \\
\bottomrule
\end{tabular}
\end{table}

\section{EXPECTED OUTCOMES AND CONTRIBUTIONS}

This methodology is expected to produce:

\begin{enumerate}
\item A comprehensive, culturally-authenticated talking drums dataset (500+ samples)
\item Novel computational methods for traditional music analysis
\item AI models capable of generating culturally-appropriate rhythmic patterns
\item Frameworks for ethical AI development in cultural contexts
\item Open-source tools for ethnomusicological research
\end{enumerate}

The research contributes to both computational musicology and cultural preservation, establishing new standards for technology-assisted cultural research.

\section{CONCLUSION}

This methodology chapter presents a comprehensive, culturally-sensitive approach to developing AI systems for traditional music analysis and generation. The framework balances technical rigor with cultural authenticity, ensuring that computational advancement serves cultural understanding and preservation.

The systematic approach, combining rigorous mathematical frameworks with respectful cultural engagement, creates a model for future research at the intersection of technology and traditional knowledge. The detailed implementation timeline and comprehensive evaluation framework ensure reproducible, reliable results that advance both scientific understanding and cultural appreciation.

\newpage
\bibliography{references}

\end{document}
